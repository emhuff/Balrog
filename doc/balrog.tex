\documentclass[12pt]{article}

\usepackage{amsfonts,amsmath,amssymb}
\usepackage{graphicx}
\usepackage{color}
\usepackage[colorlinks,linkcolor=blue,citecolor=blue,urlcolor=blue,pdftitle={Balrog},pdfauthor={Suchyta}]{hyperref}
\usepackage[margin=1.0in]{geometry}
\usepackage{longtable}
\usepackage{tabu}
\usepackage{title}
\usepackage{optstable}
\usepackage{array}


%\renewcommand{\thesection}{\arabic{section}}
\renewcommand*\sectionautorefname{Section}
\renewcommand*\subsectionautorefname{Section}
\renewcommand*\subsubsectionautorefname{Section}

\newcommand{\balrog}{Balrog}
\newcommand{\sex}{\textsc{SExtractor}}
\newcommand{\opt}[1]{\texttt{--#1}}
\newcommand{\bcmd}[1]{\vspace{6pt} \noindent \texttt{\% ./balrog.py #1} \vspace{6pt}}



\begin{document}
\balrogtitlepage

\newpage
\tableofcontents

\newpage
\section{Introduction}
\label{sec:intro}
\balrog{} is ...
%\autoref{sec:intro}

\section{Usage}
Usage

\section{Command Line Options}
\balrog{} runs can be configured via command line options.
Two types of options exist. First are the built-in
ones, native to \balrog{}. In addition, \balrog{}
supports a mechanism for users to define their
own command line options.
To print a list of all \balrog{}'s command line options,
both native and user-defined, along with
brief help strings, run:

\bcmd{\opt{help}}

\noindent \autoref{sec:builtin} further details each
of the native options and \autoref{sec:user}
explains how to create custom options.

\subsection{Built-in Options}
\label{sec:builtin}

\optstab{}

\subsection{User-defined Options}
\label{sec:user}

Within the \texttt{config.py} file, user's are able to
define their own command line options. This occurs
within the function \texttt{CustomArgs}. Passed
to \texttt{CustomArgs} as an argument is \texttt{parser},
an object made by \texttt{python}'s
\texttt{argparse.ArgumentParser()}. Arguments
can be added to parser according to the usual
\texttt{argparse} syntax.
For those unfamilar with \texttt{argparse},
\href{http://docs.python.org/2/howto/argparse.html}{this tutorial}
contains many useful examples. A simple example of
\texttt{CustomArgs} is copied below.

\setlength{\tabcolsep}{0pt}
\begin{longtabu} to \linewidth {l X}
\multicolumn{2}{l}{\texttt{def CustomArgs(parser):}}\\
\hspace{20pt} \texttt{parser.add\_argument} & \texttt{( "-cs", "--catalogsample", help="Catalog used to
sample simulated galaxy parameter distriubtions from", type=str, default=None)}
\end{longtabu}
\setlength{\tabcolsep}{6pt}


User-defined options are parsed within the function \texttt{CustomParseArgs},
also part of \texttt{config.py}.
Passed as an argument to \texttt{CustomParseArgs} is \texttt{args}, equivalent
to an object returned by \texttt{parser.parse\_args()}. Each one of the user's 
command line options becomes an attribute of \texttt{args}. 
A simple version of \texttt{CustomParseArgs} has been included below.

\setlength{\tabcolsep}{0pt}
\begin{longtabu} to \linewidth {X}
\texttt{def CustomParseArgs(args):}\\
\hspace{20pt} \texttt{thisdir = os.path.dirname( os.path.realpath(\_\_file\_\_) )} \\
\hspace{20pt} \texttt{if args.catalogsample==None:} \\
\hspace{40pt} \texttt{args.catalogsample = os.path.join(thisdir, 'cosmos.fits')}
\end{longtabu}
\setlength{\tabcolsep}{6pt}

\noindent The ability to define and parse one's own command line arguments is intended to make
\balrog{} flexible to conveniently running a wide variety of different
simulation scenarios. 

\section{Defining the Simulation}
\label{sec:simrules}

Something

\section{Output}
Each \balrog{} run generates a number of output files. 
These are organized into a fixed directory structure.
Users indicate the \opt{outdir} command line option, and
the remainder of the naming scheme occurs automatically,
placing files in subdirectories under \opt{outdir}.
Four subdirectories are written, labelled accoring to what
type of files they contain. 

\begin{itemize}
	\item \texttt{balrog\_cat}: Contains catalog files.
	\item \texttt{balorg\_image}: Contains image files.
	\item \texttt{balrog\_log}: Contains log files.
	\item \texttt{balrog\_sexconfig}: Contains files for configuring sextractor.
\end{itemize}

\end{document}
